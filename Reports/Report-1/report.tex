% File project.tex
%% Style files for ACL 2021
\documentclass[11pt,a4paper]{article}
\usepackage[hyperref]{acl2021}
\usepackage{times}
\usepackage{booktabs}
\usepackage{todonotes} \usepackage{latexsym}
\renewcommand{\UrlFont}{\ttfamily\small}

% This is not strictly necessary, and may be commented out,
% but it will improve the layout of the manuscript,
% and will typically save some space.
\usepackage{microtype}

\aclfinalcopy 

\newcommand\BibTeX{B\textsc{ib}\TeX}

\title{11-891 Report 1: Dataset Analysis and Baseline/Method Proposal}

\author{
  First Last 1\thanks{\hspace{4pt}Everyone Contributed Equally -- Alphabetical order} \hspace{2em} First Last 2$^*$ \hspace{2em} First Last 3$^*$ \hspace{2em} First Last 4$^*$ \\
  \texttt{\{ID1, ID2, ID3, ID4\}@andrew.cmu.edu}
  }

\date{}

\begin{document}
\maketitle

\section{Task Definition and Dataset Choice (1 page)}

\subsection{What phenomena or task will you work on?}

\subsection{What about this phenomena or task fundamentally involves code?}

\subsection{Does a dataset exist, or will you be collecting it, or does your task not require a dataset?}
If the dataset exists already, reference it here. If it doesn't, detail how you'll collect it and how long it will take, or justify why your task doesn't require a dataset.

\clearpage
\section{Dataset Analysis (1 page)}
\label{sec:dataset-analysis}
\subsection{Dataset properties} (GBs, code languages, numbers of examples, ...)
\subsection{Instance analysis}
(use a small sample -- e.g. validation splits):
  \begin{enumerate}
    \item Code diversity: e.g. syntactic complexity, API usage, input and output spaces
    \item If your task/dataset involves language: Lexical diversity, sentence length, ...
  \end{enumerate}
\subsection{Evaluation metrics}
What intrinsic (intermediate) and extrinsic (end-task-related) evaluations will you use to know if things are working?

\clearpage

\section{Related Work and Background}
\label{sec:related-work}
Aim for 5 papers per person, on related datasets, baselines, and/or relevant techniques.

\clearpage
\section{Baselines and Proposed Approach (1 page)}

\subsection{Proposed Baselines}
Ideally at least two baselines, although one is ok if it will be difficult to implement (e.g. no publicly released code or models).

We will ask you to have results from these, and analysis of them, in Report 2.

\subsection{Proposed Approach(es)}
What might you do beyond the baseline? What motivates this approach, based on your findings in the Dataset Analysis (Sec.\ \ref{sec:dataset-analysis})  or in Related Work (Sec.\ \ref{sec:related-work})?

\subsection{Compute requirements}
\begin{enumerate}
  \item Files (if you have a large training dataset -- can fit in RAM?)
  \item Models (can fit on GCP/AWS GPUs?)
  \item Training (if you're training models, estimated GPU hours)
  \item Inference (how many instances in your task/dataset? Do you need to sample many outputs? Execute code?)
\end{enumerate}

\clearpage
\section{Project-related}
\subsection{Comments and questions}
If you forsee any challenges, or have open qustions or want feedback on specific things, you can mention them here.

\subsection{Expertise}
We have the following expertise in the underlying aspects required by this project
  \begin{enumerate}
      \item Team member 1: Research paper in static analysis, ...
      \item Team member 2: Took NLP in Fall 2023, ...
      \item ...
  \end{enumerate}


\subsection{Team member contributions to this report}
\paragraph{Member 1} contributed ...

\paragraph{Member 2} contributed ...

\paragraph{...} contributed ...

\clearpage
% Please use 
\bibliographystyle{acl_natbib}
\bibliography{references}

%\appendix



\end{document}
